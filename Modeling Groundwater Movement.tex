\documentclass{article}

%\usepackage{tocbibind}

\usepackage[english]{babel}
\usepackage[a4paper,top=3cm,bottom=3cm,left=2cm,right=2cm,marginparwidth=1.75cm]{geometry}
\usepackage{amsmath}
\usepackage[colorlinks=true, allcolors=blue]{hyperref}

\title{
\normalfont \normalsize 
\textsc{Mathematical Modeling} \\ [20pt]
\huge Modeling Groundwater Movement: \\ Pollution and Saltwater Intrusion
}
\author{Megan Lynn Tucker}
\date{August 15, 2018}

%_____________________________________________________________________________

\setlength {\marginparwidth }{2cm}

\begin{document}

\maketitle

\pagebreak

\section*{Abstract}
Analyzing brine and saltwater movement in groundwater systems can help analyze many important environmental problems: 
disposing toxic or radioactively hazardous waste in crystalline or salt rock formations, 
leachates from landfills and industrial waste disposals infiltrating the water table\cite{1}, 
saltwater intrusion in costal aquifers pumped below sea level\cite{2} or in aquifers located above salt formations\cite{3}. 
These are all examples of groundwater pollution. 
Groundwater pollution is the deterioration of water quality resulting from human activities\cite{4}. 

\section*{Introduction}
Groundwater pollution is usually caused by one of four sources: domestic, industrial, agricultural, or environmental. 
Domestic pollution may be caused by sewers breaking, septic tanks percolating into the ground, rain infiltrating through landfills, artificially recharging aquifers using sewage water, or biological contaminants. 
Industrial pollution may also be caused by the sewage disposal system breaking. 
The difference between this and domestic pollution is that heavy metals, radioactive materials, or various non-deteriorating highly toxic compounds may also be part of the waste. 
Agricultural pollution caused by irrigation and rain water dissolving and transporting fertilizers, salts, herbicides, and pesticides into the aquifer. 
Environmental pollution is caused by the environment where the groundwater comes from.
For example, water may flow through and dissolve carbonate rocks.
Alternatively, saltwater encroachment or intrusion may occur when the equilibrium between seawater and freshwater is disturbed \cite{5}.

Saltwater encroachment can result from many things.
It occurs most often when ground-water heads are artificially lowered. 
Near the coast, fresh groundwater is discharged into the ocean or estuaries containing brackish water, a mixture of saltwater and freshwater\cite{2}. 
Many inland aquifers are bounded below by deep saltwater pockets\cite{3}. 
If saltwater intrudes into the freshwater groundwater system, the water can become permanently contaminated and unpotable for humans.

The relationship between freshwater and saltwater is primarily based on their different densities (mass per unit volume). 
The density of freshwater is approximately 1g/$cm^3$ compared to the density of saltwater, 1.025 g/$cm^3$. 
Because it is less dense, freshwater typically sits at the surface.
Under stable conditions the boundary between the two layers remains at equilibrium. 
The boundary is typically a gradient from freshwater to saltwater. 
This is called either the zone of diffusion, zone of dispersion, or transition zone. 
If an aquifer lies near saline groundwater, pumping from it will move the boundary. 
If moved far enough, the well may become contaminated with salt. 
How much the saltwater moves and at what speed is determined by the location and magnitude of groundwater withdrawals with respect to the location of the $saltwater^3$.

\section*{Main Results}
Most models for the flow of groundwater contaminants are only accurate for small amounts of contaminants. 
Oftentimes the concentration of solvents is so low the effects on density may be ignored. 
However, there are sometimes situations where groundwater with a high salt concentration is encountered. 
A model for solutions with higher concentrations of contaminants was found using a more generalized form of Darcy’s and Fick’s laws\cite{1}. 
Darcy's law is 
    \begin{align}
        Q= - \frac{KA\Delta p}{\mu L}
    \end{align}
\noindent where Q is the total discharge of fluid in $m^3$/s, K is the hydraulic conductivity of the medium measured in $m^2$, A is the cross-sectional area to flow, $\Delta$p is the total pressure drop in Pascals, $\mu$ is the viscosity in Pascal seconds, and L is the length over which the pressure drop occurs. 

Fick's first law is commonly used to describe systems containing contaminants.
It describes molecular diffusion which is caused by thermal kinetic energy causing solute molecules to have random motion. 
Fick's first law assumes steady state—the system is unchanged over time—and relates the dispersion of solutes to their concentration for two or more dimensions. 
The diffusion coefficient for a porous medium is smaller than that of a pure liquid because there are collisions with pore walls. 
Fick's law is 
    \begin{align}
        J = - D[\nabla(nC) + \frac{t}{V}] 
    \end{align} 
\noindent where V is a chemical averaging volume, n is the porosity of the medium, and t is the tortuosity of the medium. J is the diffusion flux vector which measures the amount of substance which will flow through a unit area during a unit time interval, D is the diffusion coefficient or diffusivity, C is the mass concentration of the contaminant, usually salt, and x is the position. 
Note that $\nabla$C denotes the gradient of C.
This is a generalized form of the first derivative. 
For practical application, Fick's law may be simplified to 
    \begin{align}
        J = - Dn\nabla C
    \end{align}
\noindent where D is the bulk diffusion constant. 
Oftentimes, Fick's law is further simplified to:
    \begin{align}
        J = - D\nabla C.
    \end{align}

When modeling systems of a fluid containing contaminants, these two equations are used in the following manner:
    \begin{align}
        Q = - \frac{K}{\mu}(\Delta p - \rho g)
    \end{align}
    \begin{align}
        J =- D\nabla C.
    \end{align}
\noindent Here, q is the mean velocity of the fluid, k is again the hydraulic conductivity, $\rho$ and $\mu$ are mass density and dynamic viscosity of the fluid, p is the pressure, g is the gravity vector, J is the diffusion flux of salt, D is the dispersion tensor, and C is the mass concentration of the contaminant, usually salt\cite{4}. 
The dispersion tensor is often assumed to be independent of $\nabla$C.
It is described by 
    \begin{align}
        D = (nD_{0} + \alpha \tau q)I + (\alpha^{}_{L} - \alpha^{}_{T})\frac{\textbf{qq}}{q}
    \end{align}
\noindent where n is the porosity, $D_0$ is the effective molecular diffusion coefficient,$\alpha^{}_{L}$ and$\alpha^{}_{T}$ are the longitudinal and transversal dispersivities, \textbf{q} is the apparent mean velocity of fluid with respect to the rock, q is the magnitude, and I is the unit tensor\cite{5}.

In three dimensions, Darcy's law is generalized to 
    \begin{align}
        Q = -K\nabla \varphi
    \end{align}
\noindent where $\varphi$ represents the piezometric head, the sum of the potential energy and the pressure energy per unit weight of water. 
It is defined by $\varphi$ = z + p/$\gamma$.
Here z represents the elevation of the point, p the pressure, and $\gamma$ the volumetric weight of the water\cite{6}. 
Therefore, we could rewrite the above equation as 
    \begin{align}
        Q = -K\nabla(z + \frac{p}{\gamma})
    \end{align}
\noindent where $\nabla$ (z + p/$\gamma$) is the gradient of $\varphi$.

Note that if we were modeling the motion of a fluid containing N different contaminants, in general, we would require N + 1 sets of equations\cite{4}. 
Therefore, it is prudent to assume a binary system with a single solute and a single solvent.	

\subsection*{Variations}
Let us examine the simplified equations of Darcy's and Fick's Laws:
    \begin{align}
        Q = -K\nabla\varphi
    \end{align}
    \begin{align}
        J =- D\nabla C.
    \end{align}
These equations describe motion in three dimensions.
Therefore it is prudent to simplify them to one dimension. 
So, instead of taking the gradient, which involves partial derivatives, we simply take one derivative in terms of time.
This yields
    \begin{align}
        Q = -K\frac{d}{dx}\varphi
    \end{align}
    \begin{align}
        J =- D\frac{d}{dx}C
    \end{align}
\noindent where the variables remain the same. 
The derivative is taken in terms of x, as we suppose we are in a one dimensional horizontal plane. 
We can move the derivatives to the right side of the equation and divide by the other variable, giving us
    \begin{align}
        \frac{d\varphi}{dx} = -\frac{Q}{K}
    \end{align}
    \begin{align}
        \frac{dC}{dx} = -\frac{J}{D}.
    \end{align}
\noindent Recall $\varphi$ is the sum of the potential energy and pressure energy, Q is the total discharge of the fluid, K is the hydraulic conductivity, C is the concentration, J is the chemical mass flux, and D is the diffusion constant.

Let us now show that the concentration of contaminants in the fluid neither increases nor decreases as the water flows through the ground. 
Note that here we are working with two dimensions, time and the horizontal direction. 
Again recall that $\rho$ is the mass density of the fluid. 
We have that 
    \begin{align}
        c = \int_{a}^{b} \rho (x,t) dx
    \end{align}
\noindent where a and b are two points along which the water flows in a horizontal direction. 
If the fluid flows straight and is not diverted, then the concentration between x = a and x = b may still change in time as fluid leaves at point b and enters at point a. 
If we assume no contaminant is created or destroyed in between, there can only be changes in concentration between x = a and x = b. 

We may generalize the above equation by examining situations where the sum of the potential energy and pressure energy are not constant:
    \begin{align}
        \frac{dc}{dt} = \varphi (a,t) - \varphi (b,t).
    \end{align}
\noindent Note that $\varphi$(a,t) and $\varphi$(b,t) represents the sum of the potential energy and pressure energy at some horizontal positions a and b at time t.

We may combine the two above equations into
    \begin{align}
        \int_{a}^{b} \rho (x,t) dx = \varphi (a,t) - \varphi (b,t).
    \end{align}
\noindent This equation is called a conservation law in integral form, or, an integral conservation law. 
This law expresses the idea that changes in the concentration of the contaminant are only caused the flow across the boundary between x = a and x = b.
That is, the contaminant is neither created nor destroyed and its concentration is preserved. 
However, this does not mean the concentration remains the same between x = a and x = b. 
If that were true, the integral would equal zero.

We may express the above equation as a local conservation law which is valid at any point in the water flow. 
This may be done one of three ways, but for the sake of brevity, only one will be described. 
Note that in all three derivations, we assume all functions are continuous functions of x and t. 
Furthermore, in the above equation, the full derivative must be replaced with the partial derivative with respect to time.
Besides that, the equation remains the same. 
We write the equation as
    \begin{align}
        \varphi (a,t) - \varphi (b,t) = -\int_{a}^{b} \frac{\partial}{\partial x}[\rho (x,t)] dx.
    \end{align}
\noindent We may then take the derivative with respect to b:
    \begin{align}
        \int_{a}^{b} \frac{\partial \rho (x,t)}{\partial x} + \frac{\partial \varphi (x,t)}{\partial x} dx = 0.
    \end{align}
\noindent This equation states that the definite integral is always zero regardless of the input values. This is only true for the zero function. Therefore, 
    \begin{align}
        \frac{\partial \rho}{\partial x} + \frac{\partial \varphi}{\partial x} = 0.
    \end{align}
\noindent is the local conservation law\cite{7}.

\section*{Conclusion}
Obviously, these models would be of much greater use if they were in the third dimension. 
Furthermore, we only considered the possibility of having one contaminant. 
Often, there are multiple contaminants in a water sample. 
Similarly, aquifers are rarely constructed of a single material, so the permeability of the medium likely changes as the fluid passes through the ground. 
Lastly, as the fluid travels past rocks and sediment, it can wear away the minerals.
This further contaminates the water and changes the porosity of the material. 
A good model for the movement of contaminated groundwater would contain several different variables and functions to account for this. 

Similar work may model specific instances of contaminated water flowing into an aquifer—runoff from a landfill, saltwater intrusion, or fertilizers entering the water system. 
The effects of diluting the contaminant may also be of interest.
In surface water systems, humans previously diluted the waste to the point that their affects on the ecosystems were negligible. 
Perhaps diluting a polluted aquifer enough could restore the use of that aquifer. 
The time required for fluid to percolate through the aquifer determines the time required for the contaminant to be diluted. 
Similarly, the time and concentration of contaminants required to permanently damage an aquifer may be calculated. 

\pagebreak

\bibliographystyle{amsplain}
\begin{thebibliography}{99}

\bibitem{1}
Kolditz, O., Ratke, R., Diersch, H. G., \& Zielke, W.
\textit{Coupled groundwater flow and transport: 1. Verification of variable density flow and transport models}
Advances in Water Resources, \textbf{21} No.1, (1998), pp. 27-46

\bibitem{2}
Heath, Ralph C. 
\textit{Basic ground-water hydrology}
U.S . Geological Survey Water-Supply Paper, \textbf{2220} No.86, (1983), pp. 12-26, 83

\bibitem{3}
Alley, W. M., Reilly, T. E., \& Franke, O. L. 
\textit{Sustainability of ground-water resources}
Denver, CO: U.S. Dept. of the Interior, U.S. Geological Survey.

\bibitem{4}
Hassanizadeh, S. M., \& Leijnse, T. 
\textit{On the modeling of brine transport in porous media}
Water Resources Research, \textbf{24} No.3, (1988), pp. 321-330

\bibitem{5}
Hassanizadeh, S. M., \& Leijnse, T. 
\textit{A non-linear theory of high-concentration-gradient dispersion in porous media}
Advances in Water Resources, \textbf{18} No.4, (1995), pp. 203-215

\bibitem{6}
Bear, J., \& Verruijt, A.
\textit{Modeling Groundwater Flow and Pollution (Theory and Applications of Transport in Porous Media)}
Advances in Water Resources, Dordrecht: Reidel

\bibitem{7}
Haberman, R.
\textit{Mathematical models: Mechanical vibrations, population dynamics, and traffic flow: An introduction to applied mathematics.}
Philadelphia, PA: SIAM (1998)

\end{thebibliography}

\end{document}